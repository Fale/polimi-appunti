\chapter{Matrici}
\label{Matrici} % So I can \ref{altrings} later.
\section{Somma}

\begin{equation}
A = 
\begin{pmatrix}
6 & 6 \\
6 & -1
\end{pmatrix}
B = 
\begin{pmatrix}
\sqrt{2} & -1\\
\pi & 3
\end{pmatrix}
A+B = 
\begin{pmatrix}
6+\sqrt{2} & 6-1\\
6+\pi & -1+3
\end{pmatrix}
C = 
\begin{pmatrix}
6+\sqrt{2} & 5\\
6+\pi & 2
\end{pmatrix}
\end{equation}

\subsection{Proprietà}
\begin{itemize}
 \item Gruppo Commutativo
 \begin{itemize}
  \item Associativa
  \item Neutralità dello 0
  \item Esistenza di una matrice -M
 \end{itemize}
\end{itemize}
