\documentclass[a4paper,10pt]{book}
\usepackage[utf8x]{inputenc}
\usepackage{amsmath}
\usepackage{amsfonts}
\usepackage{amssymb}
\title{Formulario di Fisica}
\author{Fabio Alessandro Locati}

\begin{document}

\chapter{Cinematica}
\section{Moto rettilineo}
\subsection{Uniforme}
\[s(t) = vt +s_0\]
\subsection{Vario}
Velocità istantanea: \[v(t)=\frac{ds}{dt}\]
\[v(t)=v_0+at\]
Accelerazione istantanea: \[a=\frac{v}{t}\]
Spazio totale: \[s=s_0+v_0t+\frac{1}{2}at^2\]
Tempo totale: \[\tau=\sqrt{\frac{2s}{a}}\]
Velocità finale: \[v_f=\sqrt{2as}\]

\section{Moto armonico}
\subsection{Semplice}
\[x(t)=A\sin{(\omega t+\varphi_0)}\]
Periodo: \[T=\frac{2\pi}{\omega}\]
Frequenza: \[f=\frac{1}{T}\]
\subsection{Smorzato esponenzialmente}
\[x(t)=Ae^{-at}\sin{(\omega t+\varphi_0)}\]

\section{Moto di caduta libera}
\subsection{Lungo la verticale}
Tempo totale: \[\tau=\sqrt{\frac{2h}{g}}\]
Velocità finale: \[v_f=\sqrt{2gh}\]
Tempo di salita: \[\tau=\frac{v_0}{g}\]
Quota massima: \[h=\frac{v_0^2}{2g}\]
\subsection{Parabolico}
Quota istantanea: \[h(t)=h_0+v_0\sin{2\alpha}t-\frac{1}{2}gt^2\]
Distanza istantanea: \[s(t) = s_0 + v_0\cos{\alpha}t\]
Tempo totale: \[\tau=2v_0\frac{\sin\alpha}{g}\]
Spazio totale (gittata): \[s(\tau)=\frac{v_0^2}{g}\sin{2\alpha}\]

\section{Moto circolare}
\subsection{Uniforme}
Velocità angolare: \[\omega=\frac{v}{R}\]
\[\omega=2\pi f\]
Periodo: \[T=\frac{2\pi}{\omega}\]
Accelerazione centripeta: \[|a|=\omega^2R\]
Angolo: \[\varTheta(t)=\varTheta_0+\omega t\]
\subsection{Vario}
Accelerazione normale: \[a_n=\omega^2R\]
Accelerazione tangenziale: \[a_t=\frac{dv}{dt}\]
\end{document}
