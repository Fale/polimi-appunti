\documentclass[a4paper,10pt]{article}
\usepackage[utf8x]{inputenc}

%opening
\title{Dinamica del punto materiale}
\author{Fabio Alesandro Locati}

\begin{document}

\maketitle

\section{Introduzione}
La dinamica studia il movimento dei corpi in relazione alle cause che lo producono. Per rendere possibile questo studio è necessario conoscere:
\begin{itemize}
 \item Le cause del moto, cioè le forze che agiscono sul corpo
 \item I parametri specifici del corpo (resistenza con l'aria, massa etc)
 \item Le equazioni del moto
\end{itemize}
La \textbf{Meccanica newtoniana} si basa su tre principi.

\section{Primo principio di Newton (o principio di inerzia)}
\textbf{In un sistema inerziale, un corpro preserva nel proprio stato di quiete o di moto rettilineo uniforme finché non agisce su esso qualche causa esterna.}
Se questo principio è evidente per i corpi in stato di quiete lo può essere meno per i corpi in stato di moto rettilineo uniforme. Questo perchè qualsiasi dimostrazione sperimentale può provare solo parzialmente questo principio, infatti sono tutte soggette all'atrito.

\section{Secondo principio di Newton}
\textbf{In un sistema di riferimento inerziale, l'accelerazione di un punto materiale è direttamente proporzionale alla forza risultante agente su esso e inversamente proporzionale alla massa del punto materiale.}
\[F=ma\]
\[[F]\equiv[m][a]\equiv[M][L][T]^{-2}\equiv[N]\]
\textbf{Un newton (1N) è l'intensità di una forza che agendo su un corpo di massa 1kg gli imprime un'accelerazione di modulo 1$\frac{m}{s^2}$}
\textbf{La forza risultante agente su un corpo e la somma vettoriale delle singole forze esercitate sul corpo dai diversi sistemi materiali che interagiscono con esso.}

\section{Terzo principio di Newton}
\textbf{Se un punto materiale esercita forza su un secondo punto materiale, quest'ultimo esercita sul primo una forza opposta}
\[F_{12}=-F_{21}\]
Da quest'ultimo principio si evince che
\[m_1a_1 = -m_2a_2\]
da cui consegue:
\[a_2=-(\frac{m_1}{m_2})a_1\]
\textbf{Quando due corpi interagiscono solo tra loro, se a un certo istante il primo corpo possiede un'accelerazione $a_1$, il secondo corpo possiede un'accelerazione $a_2=-(\frac{m_1}{m_2})a_1$}

\section{Quantità di moto}
\textbf{La quantità di moto totale di un sistema isolato rimane costante nel tempo.}

\section{Misura statica delle forze}
La dinamica viene spesso usata per determinare il peso degli oggetti. Esempi di strumenti che la usano sono il dinamometro e la bilancia.
\end{document}
